\documentclass[10pt]{article} % For LaTeX2e

% This template and styling file have been adapted from the ML Reproducibility Challenge 2020 and TMLR templates, to ease the submission.

% If you want to show the general instuctions, use the following command:
\usepackage[instructions]{atml}
% For the final version, use this one:
%\usepackage{atml}

% If you want to submit this work to tmlr, use the "tmlr" package rather than the "atml" one. Note: double-check the files needed in TMLR's submission instructions. 
% \usepackage{tmlr}
% If accepted, instead use the following line for the camera-ready submission:
%\usepackage[accepted]{tmlr}
% To de-anonymize and remove mentions to TMLR (for example for posting to preprint servers), instead use the following:
%\usepackage[preprint]{tmlr}

% Note that the following lines have been added to atml.sty:
% \usepackage[hidelinks]{hyperref}
% \newcommand*{\doi}[1]{\href{http://doi.org/#1}{doi: #1}}
% If atml.sty is not used, then these lines might need to be added to the respective styling files.
%%%%% NEW MATH DEFINITIONS %%%%%

\usepackage{amsmath,amsfonts,bm}

% Mark sections of captions for referring to divisions of figures
\newcommand{\figleft}{{\em (Left)}}
\newcommand{\figcenter}{{\em (Center)}}
\newcommand{\figright}{{\em (Right)}}
\newcommand{\figtop}{{\em (Top)}}
\newcommand{\figbottom}{{\em (Bottom)}}
\newcommand{\captiona}{{\em (a)}}
\newcommand{\captionb}{{\em (b)}}
\newcommand{\captionc}{{\em (c)}}
\newcommand{\captiond}{{\em (d)}}

% Highlight a newly defined term
\newcommand{\newterm}[1]{{\bf #1}}


% Figure reference, lower-case.
\def\figref#1{figure~\ref{#1}}
% Figure reference, capital. For start of sentence
\def\Figref#1{Figure~\ref{#1}}
\def\twofigref#1#2{figures \ref{#1} and \ref{#2}}
\def\quadfigref#1#2#3#4{figures \ref{#1}, \ref{#2}, \ref{#3} and \ref{#4}}
% Section reference, lower-case.
\def\secref#1{section~\ref{#1}}
% Section reference, capital.
\def\Secref#1{Section~\ref{#1}}
% Reference to two sections.
\def\twosecrefs#1#2{sections \ref{#1} and \ref{#2}}
% Reference to three sections.
\def\secrefs#1#2#3{sections \ref{#1}, \ref{#2} and \ref{#3}}
% Reference to an equation, lower-case.
\def\eqref#1{equation~\ref{#1}}
% Reference to an equation, upper case
\def\Eqref#1{Equation~\ref{#1}}
% A raw reference to an equation---avoid using if possible
\def\plaineqref#1{\ref{#1}}
% Reference to a chapter, lower-case.
\def\chapref#1{chapter~\ref{#1}}
% Reference to an equation, upper case.
\def\Chapref#1{Chapter~\ref{#1}}
% Reference to a range of chapters
\def\rangechapref#1#2{chapters\ref{#1}--\ref{#2}}
% Reference to an algorithm, lower-case.
\def\algref#1{algorithm~\ref{#1}}
% Reference to an algorithm, upper case.
\def\Algref#1{Algorithm~\ref{#1}}
\def\twoalgref#1#2{algorithms \ref{#1} and \ref{#2}}
\def\Twoalgref#1#2{Algorithms \ref{#1} and \ref{#2}}
% Reference to a part, lower case
\def\partref#1{part~\ref{#1}}
% Reference to a part, upper case
\def\Partref#1{Part~\ref{#1}}
\def\twopartref#1#2{parts \ref{#1} and \ref{#2}}

\def\ceil#1{\lceil #1 \rceil}
\def\floor#1{\lfloor #1 \rfloor}
\def\1{\bm{1}}
\newcommand{\train}{\mathcal{D}}
\newcommand{\valid}{\mathcal{D_{\mathrm{valid}}}}
\newcommand{\test}{\mathcal{D_{\mathrm{test}}}}

\def\eps{{\epsilon}}


% Random variables
\def\reta{{\textnormal{$\eta$}}}
\def\ra{{\textnormal{a}}}
\def\rb{{\textnormal{b}}}
\def\rc{{\textnormal{c}}}
\def\rd{{\textnormal{d}}}
\def\re{{\textnormal{e}}}
\def\rf{{\textnormal{f}}}
\def\rg{{\textnormal{g}}}
\def\rh{{\textnormal{h}}}
\def\ri{{\textnormal{i}}}
\def\rj{{\textnormal{j}}}
\def\rk{{\textnormal{k}}}
\def\rl{{\textnormal{l}}}
% rm is already a command, just don't name any random variables m
\def\rn{{\textnormal{n}}}
\def\ro{{\textnormal{o}}}
\def\rp{{\textnormal{p}}}
\def\rq{{\textnormal{q}}}
\def\rr{{\textnormal{r}}}
\def\rs{{\textnormal{s}}}
\def\rt{{\textnormal{t}}}
\def\ru{{\textnormal{u}}}
\def\rv{{\textnormal{v}}}
\def\rw{{\textnormal{w}}}
\def\rx{{\textnormal{x}}}
\def\ry{{\textnormal{y}}}
\def\rz{{\textnormal{z}}}

% Random vectors
\def\rvepsilon{{\mathbf{\epsilon}}}
\def\rvtheta{{\mathbf{\theta}}}
\def\rva{{\mathbf{a}}}
\def\rvb{{\mathbf{b}}}
\def\rvc{{\mathbf{c}}}
\def\rvd{{\mathbf{d}}}
\def\rve{{\mathbf{e}}}
\def\rvf{{\mathbf{f}}}
\def\rvg{{\mathbf{g}}}
\def\rvh{{\mathbf{h}}}
\def\rvu{{\mathbf{i}}}
\def\rvj{{\mathbf{j}}}
\def\rvk{{\mathbf{k}}}
\def\rvl{{\mathbf{l}}}
\def\rvm{{\mathbf{m}}}
\def\rvn{{\mathbf{n}}}
\def\rvo{{\mathbf{o}}}
\def\rvp{{\mathbf{p}}}
\def\rvq{{\mathbf{q}}}
\def\rvr{{\mathbf{r}}}
\def\rvs{{\mathbf{s}}}
\def\rvt{{\mathbf{t}}}
\def\rvu{{\mathbf{u}}}
\def\rvv{{\mathbf{v}}}
\def\rvw{{\mathbf{w}}}
\def\rvx{{\mathbf{x}}}
\def\rvy{{\mathbf{y}}}
\def\rvz{{\mathbf{z}}}

% Elements of random vectors
\def\erva{{\textnormal{a}}}
\def\ervb{{\textnormal{b}}}
\def\ervc{{\textnormal{c}}}
\def\ervd{{\textnormal{d}}}
\def\erve{{\textnormal{e}}}
\def\ervf{{\textnormal{f}}}
\def\ervg{{\textnormal{g}}}
\def\ervh{{\textnormal{h}}}
\def\ervi{{\textnormal{i}}}
\def\ervj{{\textnormal{j}}}
\def\ervk{{\textnormal{k}}}
\def\ervl{{\textnormal{l}}}
\def\ervm{{\textnormal{m}}}
\def\ervn{{\textnormal{n}}}
\def\ervo{{\textnormal{o}}}
\def\ervp{{\textnormal{p}}}
\def\ervq{{\textnormal{q}}}
\def\ervr{{\textnormal{r}}}
\def\ervs{{\textnormal{s}}}
\def\ervt{{\textnormal{t}}}
\def\ervu{{\textnormal{u}}}
\def\ervv{{\textnormal{v}}}
\def\ervw{{\textnormal{w}}}
\def\ervx{{\textnormal{x}}}
\def\ervy{{\textnormal{y}}}
\def\ervz{{\textnormal{z}}}

% Random matrices
\def\rmA{{\mathbf{A}}}
\def\rmB{{\mathbf{B}}}
\def\rmC{{\mathbf{C}}}
\def\rmD{{\mathbf{D}}}
\def\rmE{{\mathbf{E}}}
\def\rmF{{\mathbf{F}}}
\def\rmG{{\mathbf{G}}}
\def\rmH{{\mathbf{H}}}
\def\rmI{{\mathbf{I}}}
\def\rmJ{{\mathbf{J}}}
\def\rmK{{\mathbf{K}}}
\def\rmL{{\mathbf{L}}}
\def\rmM{{\mathbf{M}}}
\def\rmN{{\mathbf{N}}}
\def\rmO{{\mathbf{O}}}
\def\rmP{{\mathbf{P}}}
\def\rmQ{{\mathbf{Q}}}
\def\rmR{{\mathbf{R}}}
\def\rmS{{\mathbf{S}}}
\def\rmT{{\mathbf{T}}}
\def\rmU{{\mathbf{U}}}
\def\rmV{{\mathbf{V}}}
\def\rmW{{\mathbf{W}}}
\def\rmX{{\mathbf{X}}}
\def\rmY{{\mathbf{Y}}}
\def\rmZ{{\mathbf{Z}}}

% Elements of random matrices
\def\ermA{{\textnormal{A}}}
\def\ermB{{\textnormal{B}}}
\def\ermC{{\textnormal{C}}}
\def\ermD{{\textnormal{D}}}
\def\ermE{{\textnormal{E}}}
\def\ermF{{\textnormal{F}}}
\def\ermG{{\textnormal{G}}}
\def\ermH{{\textnormal{H}}}
\def\ermI{{\textnormal{I}}}
\def\ermJ{{\textnormal{J}}}
\def\ermK{{\textnormal{K}}}
\def\ermL{{\textnormal{L}}}
\def\ermM{{\textnormal{M}}}
\def\ermN{{\textnormal{N}}}
\def\ermO{{\textnormal{O}}}
\def\ermP{{\textnormal{P}}}
\def\ermQ{{\textnormal{Q}}}
\def\ermR{{\textnormal{R}}}
\def\ermS{{\textnormal{S}}}
\def\ermT{{\textnormal{T}}}
\def\ermU{{\textnormal{U}}}
\def\ermV{{\textnormal{V}}}
\def\ermW{{\textnormal{W}}}
\def\ermX{{\textnormal{X}}}
\def\ermY{{\textnormal{Y}}}
\def\ermZ{{\textnormal{Z}}}

% Vectors
\def\vzero{{\bm{0}}}
\def\vone{{\bm{1}}}
\def\vmu{{\bm{\mu}}}
\def\vtheta{{\bm{\theta}}}
\def\va{{\bm{a}}}
\def\vb{{\bm{b}}}
\def\vc{{\bm{c}}}
\def\vd{{\bm{d}}}
\def\ve{{\bm{e}}}
\def\vf{{\bm{f}}}
\def\vg{{\bm{g}}}
\def\vh{{\bm{h}}}
\def\vi{{\bm{i}}}
\def\vj{{\bm{j}}}
\def\vk{{\bm{k}}}
\def\vl{{\bm{l}}}
\def\vm{{\bm{m}}}
\def\vn{{\bm{n}}}
\def\vo{{\bm{o}}}
\def\vp{{\bm{p}}}
\def\vq{{\bm{q}}}
\def\vr{{\bm{r}}}
\def\vs{{\bm{s}}}
\def\vt{{\bm{t}}}
\def\vu{{\bm{u}}}
\def\vv{{\bm{v}}}
\def\vw{{\bm{w}}}
\def\vx{{\bm{x}}}
\def\vy{{\bm{y}}}
\def\vz{{\bm{z}}}

% Elements of vectors
\def\evalpha{{\alpha}}
\def\evbeta{{\beta}}
\def\evepsilon{{\epsilon}}
\def\evlambda{{\lambda}}
\def\evomega{{\omega}}
\def\evmu{{\mu}}
\def\evpsi{{\psi}}
\def\evsigma{{\sigma}}
\def\evtheta{{\theta}}
\def\eva{{a}}
\def\evb{{b}}
\def\evc{{c}}
\def\evd{{d}}
\def\eve{{e}}
\def\evf{{f}}
\def\evg{{g}}
\def\evh{{h}}
\def\evi{{i}}
\def\evj{{j}}
\def\evk{{k}}
\def\evl{{l}}
\def\evm{{m}}
\def\evn{{n}}
\def\evo{{o}}
\def\evp{{p}}
\def\evq{{q}}
\def\evr{{r}}
\def\evs{{s}}
\def\evt{{t}}
\def\evu{{u}}
\def\evv{{v}}
\def\evw{{w}}
\def\evx{{x}}
\def\evy{{y}}
\def\evz{{z}}

% Matrix
\def\mA{{\bm{A}}}
\def\mB{{\bm{B}}}
\def\mC{{\bm{C}}}
\def\mD{{\bm{D}}}
\def\mE{{\bm{E}}}
\def\mF{{\bm{F}}}
\def\mG{{\bm{G}}}
\def\mH{{\bm{H}}}
\def\mI{{\bm{I}}}
\def\mJ{{\bm{J}}}
\def\mK{{\bm{K}}}
\def\mL{{\bm{L}}}
\def\mM{{\bm{M}}}
\def\mN{{\bm{N}}}
\def\mO{{\bm{O}}}
\def\mP{{\bm{P}}}
\def\mQ{{\bm{Q}}}
\def\mR{{\bm{R}}}
\def\mS{{\bm{S}}}
\def\mT{{\bm{T}}}
\def\mU{{\bm{U}}}
\def\mV{{\bm{V}}}
\def\mW{{\bm{W}}}
\def\mX{{\bm{X}}}
\def\mY{{\bm{Y}}}
\def\mZ{{\bm{Z}}}
\def\mBeta{{\bm{\beta}}}
\def\mPhi{{\bm{\Phi}}}
\def\mLambda{{\bm{\Lambda}}}
\def\mSigma{{\bm{\Sigma}}}

% Tensor
\DeclareMathAlphabet{\mathsfit}{\encodingdefault}{\sfdefault}{m}{sl}
\SetMathAlphabet{\mathsfit}{bold}{\encodingdefault}{\sfdefault}{bx}{n}
\newcommand{\tens}[1]{\bm{\mathsfit{#1}}}
\def\tA{{\tens{A}}}
\def\tB{{\tens{B}}}
\def\tC{{\tens{C}}}
\def\tD{{\tens{D}}}
\def\tE{{\tens{E}}}
\def\tF{{\tens{F}}}
\def\tG{{\tens{G}}}
\def\tH{{\tens{H}}}
\def\tI{{\tens{I}}}
\def\tJ{{\tens{J}}}
\def\tK{{\tens{K}}}
\def\tL{{\tens{L}}}
\def\tM{{\tens{M}}}
\def\tN{{\tens{N}}}
\def\tO{{\tens{O}}}
\def\tP{{\tens{P}}}
\def\tQ{{\tens{Q}}}
\def\tR{{\tens{R}}}
\def\tS{{\tens{S}}}
\def\tT{{\tens{T}}}
\def\tU{{\tens{U}}}
\def\tV{{\tens{V}}}
\def\tW{{\tens{W}}}
\def\tX{{\tens{X}}}
\def\tY{{\tens{Y}}}
\def\tZ{{\tens{Z}}}


% Graph
\def\gA{{\mathcal{A}}}
\def\gB{{\mathcal{B}}}
\def\gC{{\mathcal{C}}}
\def\gD{{\mathcal{D}}}
\def\gE{{\mathcal{E}}}
\def\gF{{\mathcal{F}}}
\def\gG{{\mathcal{G}}}
\def\gH{{\mathcal{H}}}
\def\gI{{\mathcal{I}}}
\def\gJ{{\mathcal{J}}}
\def\gK{{\mathcal{K}}}
\def\gL{{\mathcal{L}}}
\def\gM{{\mathcal{M}}}
\def\gN{{\mathcal{N}}}
\def\gO{{\mathcal{O}}}
\def\gP{{\mathcal{P}}}
\def\gQ{{\mathcal{Q}}}
\def\gR{{\mathcal{R}}}
\def\gS{{\mathcal{S}}}
\def\gT{{\mathcal{T}}}
\def\gU{{\mathcal{U}}}
\def\gV{{\mathcal{V}}}
\def\gW{{\mathcal{W}}}
\def\gX{{\mathcal{X}}}
\def\gY{{\mathcal{Y}}}
\def\gZ{{\mathcal{Z}}}

% Sets
\def\sA{{\mathbb{A}}}
\def\sB{{\mathbb{B}}}
\def\sC{{\mathbb{C}}}
\def\sD{{\mathbb{D}}}
% Don't use a set called E, because this would be the same as our symbol
% for expectation.
\def\sF{{\mathbb{F}}}
\def\sG{{\mathbb{G}}}
\def\sH{{\mathbb{H}}}
\def\sI{{\mathbb{I}}}
\def\sJ{{\mathbb{J}}}
\def\sK{{\mathbb{K}}}
\def\sL{{\mathbb{L}}}
\def\sM{{\mathbb{M}}}
\def\sN{{\mathbb{N}}}
\def\sO{{\mathbb{O}}}
\def\sP{{\mathbb{P}}}
\def\sQ{{\mathbb{Q}}}
\def\sR{{\mathbb{R}}}
\def\sS{{\mathbb{S}}}
\def\sT{{\mathbb{T}}}
\def\sU{{\mathbb{U}}}
\def\sV{{\mathbb{V}}}
\def\sW{{\mathbb{W}}}
\def\sX{{\mathbb{X}}}
\def\sY{{\mathbb{Y}}}
\def\sZ{{\mathbb{Z}}}

% Entries of a matrix
\def\emLambda{{\Lambda}}
\def\emA{{A}}
\def\emB{{B}}
\def\emC{{C}}
\def\emD{{D}}
\def\emE{{E}}
\def\emF{{F}}
\def\emG{{G}}
\def\emH{{H}}
\def\emI{{I}}
\def\emJ{{J}}
\def\emK{{K}}
\def\emL{{L}}
\def\emM{{M}}
\def\emN{{N}}
\def\emO{{O}}
\def\emP{{P}}
\def\emQ{{Q}}
\def\emR{{R}}
\def\emS{{S}}
\def\emT{{T}}
\def\emU{{U}}
\def\emV{{V}}
\def\emW{{W}}
\def\emX{{X}}
\def\emY{{Y}}
\def\emZ{{Z}}
\def\emSigma{{\Sigma}}

% entries of a tensor
% Same font as tensor, without \bm wrapper
\newcommand{\etens}[1]{\mathsfit{#1}}
\def\etLambda{{\etens{\Lambda}}}
\def\etA{{\etens{A}}}
\def\etB{{\etens{B}}}
\def\etC{{\etens{C}}}
\def\etD{{\etens{D}}}
\def\etE{{\etens{E}}}
\def\etF{{\etens{F}}}
\def\etG{{\etens{G}}}
\def\etH{{\etens{H}}}
\def\etI{{\etens{I}}}
\def\etJ{{\etens{J}}}
\def\etK{{\etens{K}}}
\def\etL{{\etens{L}}}
\def\etM{{\etens{M}}}
\def\etN{{\etens{N}}}
\def\etO{{\etens{O}}}
\def\etP{{\etens{P}}}
\def\etQ{{\etens{Q}}}
\def\etR{{\etens{R}}}
\def\etS{{\etens{S}}}
\def\etT{{\etens{T}}}
\def\etU{{\etens{U}}}
\def\etV{{\etens{V}}}
\def\etW{{\etens{W}}}
\def\etX{{\etens{X}}}
\def\etY{{\etens{Y}}}
\def\etZ{{\etens{Z}}}

% The true underlying data generating distribution
\newcommand{\pdata}{p_{\rm{data}}}
% The empirical distribution defined by the training set
\newcommand{\ptrain}{\hat{p}_{\rm{data}}}
\newcommand{\Ptrain}{\hat{P}_{\rm{data}}}
% The model distribution
\newcommand{\pmodel}{p_{\rm{model}}}
\newcommand{\Pmodel}{P_{\rm{model}}}
\newcommand{\ptildemodel}{\tilde{p}_{\rm{model}}}
% Stochastic autoencoder distributions
\newcommand{\pencode}{p_{\rm{encoder}}}
\newcommand{\pdecode}{p_{\rm{decoder}}}
\newcommand{\precons}{p_{\rm{reconstruct}}}

\newcommand{\laplace}{\mathrm{Laplace}} % Laplace distribution

\newcommand{\E}{\mathbb{E}}
\newcommand{\Ls}{\mathcal{L}}
\newcommand{\R}{\mathbb{R}}
\newcommand{\emp}{\tilde{p}}
\newcommand{\lr}{\alpha}
\newcommand{\reg}{\lambda}
\newcommand{\rect}{\mathrm{rectifier}}
\newcommand{\softmax}{\mathrm{softmax}}
\newcommand{\sigmoid}{\sigma}
\newcommand{\softplus}{\zeta}
\newcommand{\KL}{D_{\mathrm{KL}}}
\newcommand{\Var}{\mathrm{Var}}
\newcommand{\standarderror}{\mathrm{SE}}
\newcommand{\Cov}{\mathrm{Cov}}
% Wolfram Mathworld says $L^2$ is for function spaces and $\ell^2$ is for vectors
% But then they seem to use $L^2$ for vectors throughout the site, and so does
% wikipedia.
\newcommand{\normlzero}{L^0}
\newcommand{\normlone}{L^1}
\newcommand{\normltwo}{L^2}
\newcommand{\normlp}{L^p}
\newcommand{\normmax}{L^\infty}

\newcommand{\parents}{Pa} % See usage in notation.tex. Chosen to match Daphne's book.

\DeclareMathOperator*{\argmax}{arg\,max}
\DeclareMathOperator*{\argmin}{arg\,min}

\DeclareMathOperator{\sign}{sign}
\DeclareMathOperator{\Tr}{Tr}
\let\ab\allowbreak


\usepackage{hyperref}
\usepackage{url}
\usepackage{tikz}
\usepackage{subcaption}
\usepackage[subpreambles=true]{standalone}
\usepackage{multirow}
\usepackage{multicol}

\title{ATML Report}

\author{%
  \name Jonatan Bella \email \href{mailto:jonatan.bella@usi.ch}{jonatan.bella@usi.ch}
  \AND
  \name Alessia Berarducci \email \href{mailto:alessia.berarducci@usi.ch}{alessia.berarducci@usi.ch}
  \AND
  \name Jonas Knupp \email \href{mailto:jonas.knupp@usi.ch}{jonas.knupp@usi.ch}
  \AND
  \name Tobias Erbacher \email \href{mailto:tobias.erbacher@usi.ch}{tobias.erbacher@usi.ch}
}

\def\groupid{Reinforcement Learning}
\def\projectid{RLE}


\begin{document}

\maketitle

\begin{abstract}
In this project we aim to analyze the claims of \cite{rle-paper} in the paper \href{https://arxiv.org/abs/2407.13755}{\textsc{Random Latent Exploration for Deep Reinforcement Learning}} which introduces a new technique to explore the state space and thus yield better overall agent scores. Moreover, we intend to reproduce the results obtained by the authors, expand on the findings and verify that the results are not cherry-picked. \textcolor{red}{<write the main findings of our project>} This report is prepared as part of the course project in \textit{Advanced Topics in Machine Learning} at \textit{Università della Svizzera italiana} in the autumn semester of 2024.
\end{abstract}

%%%%

\section{Introduction}
\noindent The paper we investigate deals with the problem of motivating an agent in a high-dimensional state space to explore the environment more exhaustively during training and thereby find non-obvious trajectories that can lead to higher long-term rewards for both discrete and continuous action spaces. The paper compares the new \textit{Random Latent Exploration} (RLE) technique to standard \textit{Proximal Policy Optimization} (PPO, see \cite{ppo-paper}), \textit{NoisyNet} (see \cite{noisynet-paper}) and \textit{Random Network Distillation} (RND, see \cite{rnd-paper}).

\noindent Exploration is one of the major issues of Reinforcement Learning (RL) and its techniques can generally be divided into two categories: Noise-based exploration and bonus-based exploration. There are advantages and disadvantages for both of them but we always have to keep in mind that always choosing the highest short-term (local) reward does not necessarily also yield the highest long-term (global) reward, e.g. picture the environment as shown in figure \ref{fig:sample-env-1}. Here, the agent starting in the \textcolor{blue}{blue state} will always choose the small reward of $1$, i.e. going right, then move back to the initial state, then move right, and so on, dithering between these two states, instead of accepting to collect a reward of $0$ by going left in order to be able to collect the much higher reward of $100$ in the following step.

\begin{figure}[h!]
  \centering
  \begin{subfigure}[b]{0.45\textwidth}
    \centering
    \includegraphics[width=\textwidth]{tikz/sample-env-1.tex}
    \caption{Pure rewards.}
    \label{fig:sample-env-1}
  \end{subfigure}
  \hfill
  \begin{subfigure}[b]{0.45\textwidth}
    \centering
    \includegraphics[width=\textwidth]{tikz/sample-env-2.tex}
    \caption{Rewards with randomization (example).}
    \label{fig:sample-env-2}
  \end{subfigure}
  
  \caption{An exemplary environment consisting of four state where transitions can occur between neighboring states. The \textcolor{blue}{blue node} is the initial state and the numbers are the rewards.}
  \label{fig:sample-env}
\end{figure}

\subsection{Random Latent Exploration}

\noindent The new idea that the authors \cite{rle-paper} present is to augment the reward function by adding a randomized term which incentivizes the agent to explore a larger portion of the state space. In particular, we will call this term $F(s, z)$ where $s \in \mathcal{S}$ is any state of the state space $\mathcal{S}$, and $\textbf{z} \in \mathbb{R}^{d}$ is a $d$-dimensional vector sampled from a given distribution $P_{\textbf{z}}$ \textcolor{red}{<elaborate on the distribution>}. So if at time step $t \in \{0, 1, 2, ..., T\}$ the agent in state $s_{t}$ takes action $a_{t} \sim \pi(. | s_{t})$ from policy $\pi$, then it will obtain the task reward $r (s_{t}, a_{t})$. In RLE, we train a so-called latent-conditioned policy network $\pi(. | s, \textbf{z})$ and latent-conditioned value network $V^{\pi}(s, \textbf{z})$ to estimate and then maximize the expected sum of rewards from a given state, aware of the random term $z$:

\begin{equation}
    V^{\pi}(s, \textbf{z}) \approx \mathbb{E}_{\pi} \left[\sum_{t = 0}^{\infty} \gamma^{t} \left(r(s_{t}, a_{t}) + F(s_{t+1}, \textbf{z})\right) \right]
    \label{eq:value_func}
\end{equation}

\noindent In \eqref{eq:value_func}, $\gamma$ describes the discount factor and $F(s_{t+1}, \textbf{z}) =  \boldsymbol{\phi}(s) \cdot \textbf{z}$ where $\boldsymbol{\phi}(s): \mathcal{S} \rightarrow \mathbb{R}^{d}$ is a feature extraction network. Both $\boldsymbol{\phi}$ and $\textbf{z}$ are $d$-dimensional vectors. Note that $z$ is resampled only before a new trajectory $(s_{0}, a_{0}, r_{0}, s_{1}, ...)$ is explored, not before each action that the agent takes. To continue the example from above, in figure \ref{fig:sample-env-2} the agent would prioritize to go left as this promises the greater reward. It is not hard to imagine that for higher dimensional environments, every time we start a new trajectory and correspondingly sample a new $z$, the agent will explore a different region of the state space and thus we can discover non-obvious paths to maximize the rewards. For the detailed pseudocode of this algorithm, see \hyperlink{algo-rle}{Appendix A} \textcolor{red}{(check link before submmission)}.

\subsection{Proximal Policy Optimization}

\noindent The algorithm constituting the foundation of RLE is Proximal Policy Iteration (PPO), described in \cite{ppo-paper}, which can itself be run to solve RL problems. It is an on-line policy gradient method where after running the policy $\boldsymbol{\pi}_{\theta_{\text{old}}}$ in parallel with $N$ actors for $T << E$ timesteps, where $E$ is the length of an episode, we compute the advantage estimator for each time index $t \in [0, T]$:

\begin{equation}
  \hat{A}_{t} = \delta_{t} + \left(\gamma \lambda\right) \delta_{t+1} + ... + \left(\gamma \lambda\right)^{T-t+1} \delta_{T - 1}
  \label{eq:ppo-advantage-estimator}
\end{equation}

\noindent where we have $\delta_{t} = r_{t} + \gamma V(s_{t+1}) - V(s_{t})$. Here, $\lambda$ is the Generalized Advantage Estimation (GAE) parameter and $V(s)$ is a learned state-value function. Afterwards, for each $t$ the surrogate objective is computed and the parameters $\theta$ are optimized e.g. with minibatch SGD or Adam (note that we optimize by maximizing the objective). This surrogate objective takes the form:

\begin{equation}
  L_{t}^{\text{CLIP} + \text{VF} + \text{S}}(\theta) = \hat{\mathbb{E}}_{t} \left[L_{t}^{\text{CLIP}}(\theta) - c_{1} L_{t}^{\text{VF}}(\theta) + c_{2} S[\boldsymbol{\pi}_{\theta}](s_{t})\right]
  \label{eq:ppo-objective-function}
\end{equation}

\noindent In \eqref{eq:ppo-objective-function}, we have the VF coefficient $c_{1}$ and the entropy coefficient $c_{2}$, together with the (optional, to enhance exploration) entropy bonus $S[\boldsymbol{\pi}_{\theta}](s_{t})$, which is calculated as the sum of $- p \log{p}$ over each of the action-probabilities $p$ resulting from the policy distribution $\boldsymbol{\pi}_{\theta}$ for a state $s_{t}$, and the objectives:

\begin{align}
  \label{eq:ppo-clip-objective} L_{t}^{\text{CLIP}}(\theta) &= \hat{\mathbb{E}}_{t}\left[\min\left(r_{t}(\theta) \hat{A}_{t}, \text{clip}\left(r_{t}(\theta), 1 - \epsilon, 1 + \epsilon\right) \hat{A}_{t}\right)\right]\\
  \label{eq:ppo-vf-objective} L_{t}^{\text{VF}}(\theta) &= \left(V_{\theta}(s_{t}) - V_{t}^{\text{targ}}\right)^{2}
\end{align}

\noindent In \eqref{eq:ppo-clip-objective} we denote the probability ratio as $r_{t}(\theta) = \frac{\pi_{\theta}\left(a_{t}|s_{t}\right)}{\pi_{\theta_{\text{old}}}(a_{t}|s_{t})}$, where $\pi_{\theta_{\text{old}}}$ is the policy before the update, and the clipping function is:

\begin{equation}
  \text{clip}\left(r_{t}(\theta), 1 - \epsilon, 1 + \epsilon\right) = \begin{cases}
    1 - \epsilon, &r_{t}(\theta) < 1 - \epsilon\\
    r_{t}(\theta), &1 - \epsilon \leq r_{t}(\theta) \leq 1 + \epsilon\\
    1 + \epsilon, &r_{t}(\theta) > 1 + \epsilon
  \end{cases}
  \label{eq:clipping-function}
\end{equation}

\noindent Moreover in \eqref{eq:ppo-vf-objective}, $V_{t}^{\text{targ}}$ represents some target state-value function that is to be obtained. Note that there exist also other objective functions mentioned in \cite{ppo-paper} that can substitute $L_{t}^{\text{CLIP}}(\theta)$ in \eqref{eq:ppo-objective-function}. In \eqref{eq:clipping-function}, we make use of a hyperparameter $\epsilon \in [0, 1]$ and in combination with the minimizer from \eqref{eq:ppo-clip-objective} we can prevent a single policy update to accidentally ruin the policy forever by sending the algorithm to a gradient region with a worse local extremum. The detailed pseudocode of this algorithm can be found in \hyperlink{algo-ppo}{Appendix B} \textcolor{red}{(check link before submmission)}.

\noindent In the paper that we investigate, the authors \cite{rle-paper} compare the empirical results of the RLE to PPO as the baseline, as well as two other exploration techniques, namely NoisyNet and RND, which will be briefly explained in the following sections.

\subsection{NoisyNet}

\noindent The central idea in a NoisyNet architecture brought forward by \cite{noisynet-paper} is to introduce random noise into the weights learned by the network ($p$ inputs, $q$ outputs), i.e. if we have a neural network whose output can be parametrized by a vector of weights $\boldsymbol{\theta}$ as $\textbf{y} = f_{\boldsymbol{\theta}}(\textbf{x}) = \textbf{w} \cdot \textbf{x} + \textbf{b}$, then we can learn the set of parameter vectors $\zeta = (\boldsymbol{\mu}, \boldsymbol{\Sigma})$ to compute $\textbf{w} = (\boldsymbol{\mu}^{w} + \boldsymbol{\sigma}^{w} \odot \boldsymbol{\varepsilon}^{w}$ and $\textbf{b} = (\boldsymbol{\mu}^{b} + \boldsymbol{\sigma}^{b} \odot \boldsymbol{\varepsilon}^{b})$ with the following dimensionalities: $\boldsymbol{\mu}^{w}$, $\boldsymbol{\sigma}^{w}$, $\boldsymbol{\varepsilon}^{w} \in \mathbb{R}^{q \times p}$ implying $\textbf{w} \in \mathbb{R}^{q \times p}$, and $\boldsymbol{\mu}^{b}$, $\boldsymbol{\sigma}^{b}$, $\boldsymbol{\varepsilon}^{b} \in \mathbb{R}^{q}$ resulting in $\textbf{b} \in \mathbb{R}^{q}$. Note that $\textbf{x} \in \mathbb{R}^{p}$, $\textbf{y} \in \mathbb{R}^{q}$ and $\odot$ denotes element-wise multiplication.

\noindent We optimize the loss $\bar{L}(\zeta) = \mathbb{E}_{\varepsilon}[L(\boldsymbol{\theta})]$, i.e. the expectation of the loss $L(\boldsymbol{\theta})$ over the noise $\varepsilon$ which has a mean of zero and fixed statistics, using gradient descent on the parameters $\zeta$. The noise can be sampled either independently or in a factorized way from a Gaussian. The former method samples the noise $\varepsilon^{w}_{i,j} \in \boldsymbol{\varepsilon}^{w}$ and $\varepsilon^{b}_{j} \in \boldsymbol{\varepsilon}^{b}$ for every input to the network leading to a total of $pq + q$ random samples. Factorized sampling on the other hand greatly reduces computational complexity by factorizing $\varepsilon^{w}_{i,j} = f(\varepsilon_{i}) f(\varepsilon_{j})$ and then $\varepsilon^{b}_{j} = f(\varepsilon_{j})$ thereby lowering the number of required random samples to $p + q$, where $f: \mathbb{R} \rightarrow \mathbb{R}$. In \cite{noisynet-paper} the function $f(x) = \sign(x) \sqrt{|x|}$ is used. After sampling the noise, in theory we compute the gradients $\nabla\bar{L}(\zeta) = \nabla\mathbb{E}_{\varepsilon}[L(\boldsymbol{\theta})] = \mathbb{E}_{\varepsilon}[\nabla_{\boldsymbol{\mu}, \boldsymbol{\Sigma}}L(\boldsymbol{\mu} + \boldsymbol{\Sigma} \odot \boldsymbol{\varepsilon})]$, whereas in practice we approximate $\nabla\bar{L}(\zeta) \approx \nabla_{\boldsymbol{\mu}, \boldsymbol{\Sigma}}L(\boldsymbol{\mu} + \boldsymbol{\Sigma} \odot \boldsymbol{\varepsilon})$ with the Monte Carlo method.

\noindent In the code of \cite{rle-paper}, a NoisyNet is used in conjunction with the A3C-algorithm, in particular the loss is computed from the action-value function is estimated using:

\begin{equation}
  \hat{Q}_{i} = \sum_{j=i}^{k-1} \gamma^{j-i} r_{t+j} + \gamma^{k-i} V(x_{t+k} | \zeta, \varepsilon_{i})
\end{equation}

Moreover, for factorized networks, we initialize $\mu_{i,j} \sim \mathcal{U}\left[-\frac{1}{\sqrt{p}}, +\frac{1}{\sqrt{p}}\right]$ and $\sigma_{i,j} = \frac{\sigma_{0}}{\sqrt{p}}$ with $p$ being the number of inputs to the corresponding linear layer, $\mathcal{U}$ is a uniform distribution over the specified interval, and $\sigma_{0}$ is a hyperparameter. The pseudocode for this variant given by \cite{noisynet-paper} can be found in appendix \hyperlink{algo-noisynet}{Appendix C} \textcolor{red}{Check link before submission}.

\textcolor{red}{Are we using independent or factorized sampling in the code? (I'm guessing factorized).}

\subsection{Random Network Distillation}

In Random Network Distillation, as described by \cite{rnd-paper}, we compose the reward which the agent obtains as $r_{t} = e_{t} + i_{t}$ where $e_{t}$ represents a sparse extrinsic (environment's) reward and $i_{t}$ is the intrinsic reward for transitioning, i.e. the exploration bonus emanating from the transition at time step $t$. The latter measures the novelty of a state and should yield a higher value, the less frequently a state has been visited so far. In case of a finite state space we can use $i_{t} = \frac{1}{n_{t}}$ or $i_{t} = \frac{1}{\sqrt{s}}$ where $n_{t}(s)$ denotes the number of times state $s$ has been visited until time step $t$. However, there exist alternatives, e.g. we could use state density based approaches to calculate an exploration bonus, which are described in the paper mentioned above. Moreover, they specify that in the paper the intrinsic reward is the prediction error of a randomly generated problem $i_{t} = \|\hat{f}(x|\theta) - f(x)\|^{2}$ involving a target network $f: \mathcal{O} \rightarrow \mathbb{R}^{k}$ (fixed and randomly initialized, maps an observation $\mathcal{O}$ to an embedding $\mathbb{R}^{k}$) to sample the problem as well as a predictor network $\hat{f}: \mathcal{O} \rightarrow \mathbb{R}^{k}$ which was trained on the agent's data by gradient descent to optimize $i_{t}$ w.r.t. parameters $\theta_{\hat{f}}$.

\noindent Since the overall return can be composed as a sum of returns $R = R_{E} + R_{I}$ we can train two heads $V_{E}$ and $V_{I}$ to estimate the value function $V = V_{E} + V_{I}$ and in extension the advantages. In the end, a policy is trained with standard PPO using the advantage estimators. Note that the intrinsic rewards have to be normalized in order to be useful because otherwise we could not guarantee that they are on a consistent scale, which is done by dividing $i_{t}$ by a running estimate of the standard deviations of the intrinsic return. Furthermore, the observation also has to be normalized as $x \leftarrow \text{clip}(\frac{x - \mu}{\sigma}, -5, 5)$. The normalization parameters can be initialized by letting a random agent briefly move in the environment before beginning the optimization. The pseudocode given by \cite{rnd-paper} can be found in appendix \hyperlink{algo-rnd}{Appendix D} \textcolor{red}{Check link before submission}.

\section{Scope of reproducibility}
\label{sec:claims}

\texttt{Introduce the specific setting or problem addressed in this work, and list the main claims from the original paper. Think of this as writing out the main contributions of the original paper. Each claim should be relatively concise; some papers may not clearly list their claims, and one must formulate them in terms of the presented experiments.}

\texttt{A claim should be something that can be supported or rejected by your data. An example is, ``Finetuning pre-trained BERT on dataset X will have higher accuracy than an LSTM trained with GloVe embeddings.''}
\texttt{This is concise and is something that can be supported by experiments.}
\texttt{An example of a claim that is too vague, and can't be supported by experiments, is ``Contextual embedding models have shown strong performance on a number of tasks. We will run experiments evaluating two types of contextual embedding models on datasets X, Y, and Z."}

\texttt{This section roughly tells a reader what to expect in the rest of the report. Clearly itemize the claims you are testing:}
\begin{itemize}
    \item \texttt{Claim 1}
    \item \texttt{Claim 2}
    \item \texttt{Claim 3}
    \item \texttt{Claim 4}
\end{itemize}

\texttt{Each experiment in Section~\ref{sec:results} will support (at least) one of these claims, so a reader of your report should be able to separately understand the \emph{claims} and the \emph{evidence} that supports them.}

\section{Methodology}
\texttt{Explain your approach - did you use the author's code, or did you aim to re-implement the approach from the description in the paper? Summarize the resources (code, documentation, GPUs) that you used.}

\subsection{Model descriptions}
\texttt{Include a description of each model or algorithm used. Be sure to list the type of model, the number of parameters, and other relevant info (e.g. if it's pre-trained).}

\subsection{Datasets}

\noindent Since we are dealing with a reinforcement learning task, we did not use any dataset to train our models but rather four different environments which will explain in \hyperlink{experimental-setup}{section 3.4} \textcolor{red}{(check link before submission)}.

\subsection{Hyperparameters}
\texttt{Describe how the hyperparameter values were set and what was the source for their value (e.g. paper, code, or your guess). If there was a hyperparameter search done, be sure to include the range of hyperparameters searched over, the method used to search (e.g. manual search, random search, grid search, etc.), and the best hyperparameters found. Include the number of total experiments (e.g. hyperparameter trials). You can also include all results from that search (not just the best-found results).}

\hypertarget{experimental-setup}{\subsection{Experimental setup and code}}

\texttt{Include a description of how the experiments were set up that's clear enough a reader could replicate the setup.}
\texttt{Include a description of the specific measure used to evaluate the experiments (e.g. accuracy, precision@K, BLEU score, etc.).}
\texttt{Provide a link to your code or notebook (if available). Add in this section (or in the following) any reference to cloud-based providers you}

\subsection{Computational requirements}
\texttt{Include a description of the hardware used, such as the GPU or CPU the experiments were run on. 
For each model, include a measure of the average runtime (e.g. average time to predict labels for a given validation set with a particular batch size).
For each experiment, include the total computational requirements (e.g. the total GPU hours spent).}\\
\texttt{Note: You'll likely have to record this as you run your experiments, so it's better to think about it ahead of time.}

\section{Results}
\label{sec:results}
\texttt{Start with a high-level overview of your results. Do your results support the main claims of the original paper? Keep this section as factual and precise as possible, and reserve your judgment and discussion points for the next "Discussion" section.}


\subsection{Results reproducing original paper}
\texttt{For each experiment, say 1) which claim in Section~\ref{sec:claims} it supports, and 2) if it successfully reproduced the associated experiment in the original paper.}
\texttt{For example, an experiment training and evaluating a model on a dataset may support a claim that that model outperforms some baseline.}
\texttt{Logically group related results into subsections.}

\subsubsection{Result 1}
\subsubsection{Result 2}

\subsection{Results beyond original paper}
\texttt{Often papers don't include enough information to fully specify their experiments, so some additional experimentation may be necessary. For example, it might be the case that batch size was not specified, and so different batch sizes need to be evaluated to reproduce the original results. Include the results of any additional experiments here. Note: this won't be necessary for all reproductions.}
 
\subsubsection{Additional Result 1}
\subsubsection{Additional Result 2}

\section{Discussion}

\texttt{Give your judgment on if your experimental results support the claims of the paper. Discuss the strengths and weaknesses of your approach - perhaps you didn't have time to run all the experiments, or perhaps you did additional experiments that further strengthened the claims in the paper.}

\subsection{What was easy}
\texttt{Give your judgment of what was easy to reproduce. Perhaps the author's code is clearly written and easy to run, so it was easy to verify the majority of original claims. Or, the explanation in the paper was really easy to follow and put into code.}

\texttt{Be careful not to give sweeping generalizations. Something that is easy for you might be difficult for others. Put what was easy in context and explain why it was easy (e.g. code had extensive API documentation and a lot of examples that matched experiments in papers).}

\subsection{What was difficult}
\texttt{List part of the reproduction study that took more time than you anticipated or you felt was difficult.}

\texttt{Be careful to put your discussion in context. For example, don't say "The maths was difficult to follow", say "The math requires advanced knowledge of calculus to follow".}

\subsection{Communication with original authors}
\texttt{Document the extent of (or lack of) communication with the original authors. To make sure the reproducibility report is a fair assessment of the original research we recommend getting in touch with the original authors. We advise you to contact them through their email displayed in the paper and by asking specific questions.}

\section{Conclusion}
\texttt{Try to summarize the achievements of your project and its limits, suggesting (when appropriate) possible extensions and future works.}

\section*{Member contributions}
\texttt{Include a section specifying how you organized the work within the group and clearly describe the contributions of each member. Make sure to properly balance the workload among the members.}

%%%%

\bibliography{main}
\bibliographystyle{tmlr}

\appendix
\section{Appendix}
You may include other additional sections here. Here we provide suggestions and instructions for the layout and styling of the report.

\section{Citations, figures, and tables}

\paragraph{Citations}

When the authors or the publication are
included in the sentence, the citation should not be in parenthesis, using \verb|\citet{}| (as
in ``See \citet{Hinton06} for more information.''). Otherwise, the citation
should be in parenthesis using \verb|\citep{}| (as in ``Deep learning shows promise to make progress
towards AI~\citep{Bengio+chapter2007}.'').

\paragraph{Figures}

All artwork must be neat, clean, and legible. The figure number and caption always appear after the figure. Place one line space before the figure caption, and one line space after the figure. The figure caption is lowercase (except for the first word and proper nouns). Make sure the figure caption does not get separated from the figure. Leave sufficient space to avoid splitting the figure and figure caption.

\begin{figure}[ht]
\begin{center}
\fbox{\rule[-.5cm]{0cm}{4cm} \rule[-.5cm]{4cm}{0cm}}
\end{center}
\caption{Sample figure caption.}
\end{figure}

\paragraph{Tables}

All tables must be centered, neat, clean and legible. The table number and title always appear before the table. See Table~\ref{sample-table}. Place one line space before the table title, one line space after the table title, and one line space after the table. The table title must be lowercase (except for the first word and proper nouns).

\begin{table}[ht]
\caption{Sample table title}
\label{sample-table}
\begin{center}
\begin{tabular}{ll}
\multicolumn{1}{c}{\bf PART}  &\multicolumn{1}{c}{\bf DESCRIPTION}
\\ \hline \\
Dendrite         &Input terminal \\
Axon             &Output terminal \\
Soma             &Cell body (contains cell nucleus) \\
\end{tabular}
\end{center}
\end{table}

\section{Math Notation}

In an attempt to encourage standardized notation, we have included the
notation file from the textbook, \textit{Deep Learning}
\cite{goodfellow2016deep} available at
\url{https://github.com/goodfeli/dlbook_notation/}.  Use of this style
is not required and can be disabled by commenting out
\texttt{math\_commands.tex}.
\clearpage

\centerline{\bf Numbers and Arrays}
\bgroup
\def\arraystretch{1.5}
\begin{tabular}{p{1in}p{3.25in}}
$\displaystyle a$ & A scalar (integer or real)\\
$\displaystyle \va$ & A vector\\
$\displaystyle \mA$ & A matrix\\
$\displaystyle \tA$ & A tensor\\
$\displaystyle \mI_n$ & Identity matrix with $n$ rows and $n$ columns\\
$\displaystyle \mI$ & Identity matrix with dimensionality implied by context\\
$\displaystyle \ve^{(i)}$ & Standard basis vector $[0,\dots,0,1,0,\dots,0]$ with a 1 at position $i$\\
$\displaystyle \text{diag}(\va)$ & A square, diagonal matrix with diagonal entries given by $\va$\\
$\displaystyle \ra$ & A scalar random variable\\
$\displaystyle \rva$ & A vector-valued random variable\\
$\displaystyle \rmA$ & A matrix-valued random variable\\
\end{tabular}
\egroup
\vspace{0.25cm}

\centerline{\bf Sets and Graphs}
\bgroup
\def\arraystretch{1.5}

\begin{tabular}{p{1.25in}p{3.25in}}
$\displaystyle \sA$ & A set\\
$\displaystyle \R$ & The set of real numbers \\
$\displaystyle \{0, 1\}$ & The set containing 0 and 1 \\
$\displaystyle \{0, 1, \dots, n \}$ & The set of all integers between $0$ and $n$\\
$\displaystyle [a, b]$ & The real interval including $a$ and $b$\\
$\displaystyle (a, b]$ & The real interval excluding $a$ but including $b$\\
$\displaystyle \sA \backslash \sB$ & Set subtraction, i.e., the set containing the elements of $\sA$ that are not in $\sB$\\
$\displaystyle \gG$ & A graph\\
$\displaystyle \parents_\gG(\ervx_i)$ & The parents of $\ervx_i$ in $\gG$
\end{tabular}
\egroup
\vspace{0.25cm}

\clearpage
\centerline{\bf Indexing}
\bgroup
\def\arraystretch{1.5}

\begin{tabular}{p{1.25in}p{3.25in}}
$\displaystyle \eva_i$ & Element $i$ of vector $\va$, with indexing starting at 1 \\
$\displaystyle \eva_{-i}$ & All elements of vector $\va$ except for element $i$ \\
$\displaystyle \emA_{i,j}$ & Element $i, j$ of matrix $\mA$ \\
$\displaystyle \mA_{i, :}$ & Row $i$ of matrix $\mA$ \\
$\displaystyle \mA_{:, i}$ & Column $i$ of matrix $\mA$ \\
$\displaystyle \etA_{i, j, k}$ & Element $(i, j, k)$ of a 3-D tensor $\tA$\\
$\displaystyle \tA_{:, :, i}$ & 2-D slice of a 3-D tensor\\
$\displaystyle \erva_i$ & Element $i$ of the random vector $\rva$ \\
\end{tabular}
\egroup
\vspace{0.25cm}


\centerline{\bf Calculus}
\bgroup
\def\arraystretch{1.5}
\begin{tabular}{p{1.25in}p{3.25in}}
% NOTE: the [2ex] on the next line adds extra height to that row of the table.
% Without that command, the fraction on the first line is too tall and collides
% with the fraction on the second line.
$\displaystyle\frac{d y} {d x}$ & Derivative of $y$ with respect to $x$\\ [2ex]
$\displaystyle \frac{\partial y} {\partial x} $ & Partial derivative of $y$ with respect to $x$ \\
$\displaystyle \nabla_\vx y $ & Gradient of $y$ with respect to $\vx$ \\
$\displaystyle \nabla_\mX y $ & Matrix derivatives of $y$ with respect to $\mX$ \\
$\displaystyle \nabla_\tX y $ & Tensor containing derivatives of $y$ with respect to $\tX$ \\
$\displaystyle \frac{\partial f}{\partial \vx} $ & Jacobian matrix $\mJ \in \R^{m\times n}$ of $f: \R^n \rightarrow \R^m$\\
$\displaystyle \nabla_\vx^2 f(\vx)\text{ or }\mH( f)(\vx)$ & The Hessian matrix of $f$ at input point $\vx$\\
$\displaystyle \int f(\vx) d\vx $ & Definite integral over the entire domain of $\vx$ \\
$\displaystyle \int_\sS f(\vx) d\vx$ & Definite integral with respect to $\vx$ over the set $\sS$ \\
\end{tabular}
\egroup
\vspace{0.25cm}

\centerline{\bf Probability and Information Theory}
\bgroup
\def\arraystretch{1.5}
\begin{tabular}{p{1.25in}p{3.25in}}
$\displaystyle P(\ra)$ & A probability distribution over a discrete variable\\
$\displaystyle p(\ra)$ & A probability distribution over a continuous variable, or over
a variable whose type has not been specified\\
$\displaystyle \ra \sim P$ & Random variable $\ra$ has distribution $P$\\% so thing on left of \sim should always be a random variable, with name beginning with \r
$\displaystyle  \E_{\rx\sim P} [ f(x) ]\text{ or } \E f(x)$ & Expectation of $f(x)$ with respect to $P(\rx)$ \\
$\displaystyle \Var(f(x)) $ &  Variance of $f(x)$ under $P(\rx)$ \\
$\displaystyle \Cov(f(x),g(x)) $ & Covariance of $f(x)$ and $g(x)$ under $P(\rx)$\\
$\displaystyle H(\rx) $ & Shannon entropy of the random variable $\rx$\\
$\displaystyle \KL ( P \Vert Q ) $ & Kullback-Leibler divergence of P and Q \\
$\displaystyle \mathcal{N} ( \vx ; \vmu , \mSigma)$ & Gaussian distribution %
over $\vx$ with mean $\vmu$ and covariance $\mSigma$ \\
\end{tabular}
\egroup
\vspace{0.25cm}

\clearpage
\centerline{\bf Functions}
\bgroup
\def\arraystretch{1.5}
\begin{tabular}{p{1.25in}p{3.25in}}
$\displaystyle f: \sA \rightarrow \sB$ & The function $f$ with domain $\sA$ and range $\sB$\\
$\displaystyle f \circ g $ & Composition of the functions $f$ and $g$ \\
  $\displaystyle f(\vx ; \vtheta) $ & A function of $\vx$ parametrized by $\vtheta$.
  (Sometimes we write $f(\vx)$ and omit the argument $\vtheta$ to lighten notation) \\
$\displaystyle \log x$ & Natural logarithm of $x$ \\
$\displaystyle \sigma(x)$ & Logistic sigmoid, $\displaystyle \frac{1} {1 + \exp(-x)}$ \\
$\displaystyle \zeta(x)$ & Softplus, $\log(1 + \exp(x))$ \\
$\displaystyle || \vx ||_p $ & $\normlp$ norm of $\vx$ \\
$\displaystyle || \vx || $ & $\normltwo$ norm of $\vx$ \\
$\displaystyle x^+$ & Positive part of $x$, i.e., $\max(0,x)$\\
$\displaystyle \1_\mathrm{condition}$ & is 1 if the condition is true, 0 otherwise\\
\end{tabular}
\egroup
\vspace{0.25cm}


\end{document}
